\documentclass[english]{article}
\usepackage[T1]{fontenc}
\usepackage[latin9]{inputenc}
\usepackage{color}
\usepackage{amsmath}
\usepackage{amssymb}
\usepackage{esint}
\usepackage{babel}
\usepackage[round]{natbib}
\usepackage{color,hyperref}
\definecolor{darkblue}{rgb}{0.0,0.0,0.3}
\hypersetup{colorlinks,breaklinks,
linkcolor=darkblue,urlcolor=darkblue,
anchorcolor=darkblue,citecolor=darkblue}
\usepackage[capitalise, noabbrev]{cleveref}

\begin{document}
	\title{Solving PDEs with Backward Implicit Time Steps}
	\author{\large{\textsc{Matthieu Gomez \thanks{I thank Valentin Haddad, Ben Moll, and Dejanir Silva for useful discussions.}}}}
	\date{\today}
	\maketitle
	This package \href{https://github.com/matthieugomez/EconPDEs.jl}{EconPDEs.jl} introduces a fast and robust way to solve systems of PDEs + algebraic equations (i.e. DAEs) associated with economic models. It builds on the method presented in \citet{achdou2014heterogeneous}, but extends it to handle non-linearities.

	Consider a PDE of the form
	\begin{align}\label{pde}
		0&=f_0(x, V)  + f_1(x, \partial_x V) \partial_x V  + f_2(x, \partial_x V) \partial_{xx} V +  \partial_t V
	\end{align}

	As in \citet{achdou2014heterogeneous}, I first discretize the state $x$ on a grid and I approximate derivatives $\partial_x V_t$ and $\partial_{xx} V_t$ using finite difference approximations. First order derivatives are upwinded, which helps with convergence and ensures that boundary counditions are satisfied.

	Given $V_{t+1}$, I solve for $V_t$ using a backward implicit time step:
	\begin{align}\label{fs1}
		0&=f_0(x, V_t)  + f_1(x, \partial_x V_t) \partial_x V_t  + f_2(x, \partial_x V_t) \partial_{xx} V_t +   \frac{1}{\Delta}(V_{t+1} -V_{t})
	\end{align}
	This non-linear equation is solved using a Newton-Raphson method. 	Julia packages ForwardDiff and SparseDiffTools are used to compute the (sparse) Jacobian of (\ref{fs1}) automatically.



	\paragraph{Difference with \citet{achdou2014heterogeneous}}  \citet{achdou2014heterogeneous} propose to solve the PDE (\ref{pde}) using a backward \emph{semi}-implicit time steps of the form:
	\begin{align}\label{fs2}
		0&=f_0(x, V_t)  + f_1(x, \partial_x V_{t+1}) \partial_x V_t  + f_2(x, \partial_x V_{t+1}) \partial_{xx} V_t +  \frac{1}{\Delta}(V_{t+1} -V_{t})
	\end{align}
	Compared to (\ref{fs1}), each time step corresponds to a \emph{linear} equation in $V_t$. This makes it easier to solve each step, since one only needs to invert a matrix. However, this simplification makes the method less robust: one can show that the associated scheme is typically non monotonous, unless $f_1$ and $f_2$ do not depend on the value function.


	\paragraph{Stationary Solution} In most cases, one is only interested in the stationary solution of the PDE (\ref{pde}), i.e.,
	\begin{align}\label{pde2}
		0&=f_0(x, V)  + f_1(x, \partial_x V) \partial_x V  + f_2(x, \partial_x V) \partial_{xx} V 
	\end{align}
	In this case, I use the same method, but I adapt the time step $\Delta$ over time. More precisely, if the backward time step (\ref{fs1}) is successful (i.e., the Newton-Raphson method converges), $\Delta$ is increased; otherwise, it is decreased. This method ensures converges since, the Newton-Raphson method always converges for $\Delta$ small enough. Yet, it does not sacrifice speed: as $\Delta \to \infty$, the method becomes equivalent to a non-linear solver for the PDE, which ensures quadratic convergence around the solution. 

	The idea of adapting $\Delta$ comes from the Pseudo-Transient Continuation method used in the fluid dynamics literature. Formal conditions for the convergence of the algorithm are given in  \citet{kelley1998convergence}. \par

	\paragraph{Applications} Empirically, I find the method to be fast and robust --- the examples folder shows that the algorithm solves a wide range of asset pricing models.
	\bibliography{bib}
	\bibliographystyle{aer}
\end{document}
