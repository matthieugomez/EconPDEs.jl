\documentclass[english]{article}
\usepackage[T1]{fontenc}
\usepackage[latin9]{inputenc}
\usepackage{color}
\usepackage{amsmath}
\usepackage{amssymb}
\usepackage{esint}
\usepackage{babel}
\usepackage[round]{natbib}
\usepackage{color,hyperref}
\definecolor{darkblue}{rgb}{0.0,0.0,0.3}
\hypersetup{colorlinks,breaklinks,
linkcolor=darkblue,urlcolor=darkblue,
anchorcolor=darkblue,citecolor=darkblue}
\usepackage[capitalise, noabbrev]{cleveref}

\begin{document}
	\title{Solving PDEs with Backward Implicit Time Steps}
	\author{\large{\textsc{Matthieu Gomez \thanks{I thank Valentin Haddad, Ben Moll, and Dejanir Silva for useful discussions.}}}}
	\date{\today}
	\maketitle
	This package \href{https://github.com/matthieugomez/EconPDEs.jl}{EconPDEs.jl} introduces a fast and robust way to solve systems of PDEs + algebraic equations (i.e. DAEs) associated with economic models. It builds on the method presented in \citet{achdou2014heterogeneous}, but extends it to handle non-linearities.

	Consider a PDE of the form
	\begin{align}
		\label{pde}
		\partial_t V&=f(x, V, \partial_x V, \partial_{xx} V)
	\end{align}

	As in \citet{achdou2014heterogeneous}, I first discretize the state $x$ on a grid and I approximate derivatives $\partial_x V_t$ and $\partial_{xx} V_t$ using finite difference approximations. First order derivatives are upwinded, which helps with convergence and ensures that boundary counditions are satisfied.

	This discretization allows one to rewrite the PDE as an ODE in a vector $V$:
	\begin{align*}
		\partial_t V&= F(V)
	\end{align*}
	where $F$ is a non linear function of the vector $V$.

	I then use a stiff method to solve for this  ODE. More precisely, given $V_t$, I solve for
	\begin{align}
		\frac{V_{t+1}-V_t}{\Delta}&=F(V_{t+1}),\label{fs1}
	\end{align}
	which is a non linear equation. Two solutions:
	\begin{enumerate}
		\item Solve for the non-linear equation using the Ralphson-Newson algorithm
		\item Decompose the ODE into a linear and a non linear part, and only solve for the linear part implicitly:
		\begin{align*}
			\frac{V_{t+1}-V_t}{\Delta}&= \partial_V F(V_{t}) V_{t+1} + F(V_t) - \partial_V F(V_t) V_t\\
			\Rightarrow V_{t+1} &= (1 - \Delta \partial_V F(V_{t}))^{-1}\left(V_t + \Delta (F(V_t) - \partial_V F(V_t) V_t)\right)
		\end{align*}
		In the particular case in which $F$ is linear, this becomes
			\begin{align*}
				V_{t+1} &= (1 - \Delta \partial_V F(V_{t}))^{-1}V_t
			\end{align*}
			which is the implicit method discussed in \citet{achdou2014heterogeneous}.
	\end{enumerate}
	
	In both cases, the (sparse) Jabobian $\partial_V F$ 	is automatically computed using the Julia packages ForwardDiff and SparseDiffTools. 



	\paragraph{Stationary Solution} In most cases, one is only interested in the stationary solution of the PDE (\ref{pde}), i.e.,
	\begin{align}
		\label{pde2}
		0&=f(x, V, \partial_x V, \partial_{xx} V)
	\end{align}
	In this case, I use the same method, but I adapt the time step $\Delta$ over time. More precisely, if the time iteratio (\ref{fs1}) is successful (i.e., the Newton-Raphson method converges), $\Delta$ is increased; otherwise, it is decreased. This method ensures converges since, the Newton-Raphson method always converges for $\Delta$ small enough. Yet, it does not sacrifice speed: as $\Delta \to \infty$, the method becomes equivalent to a non-linear solver for the PDE, which ensures quadratic convergence around the solution. 

	The idea of adapting $\Delta$ comes from the Pseudo-Transient Continuation method used in the fluid dynamics literature. Formal conditions for the convergence of the algorithm are given in  \citet{kelley1998convergence}. \par

	\paragraph{Applications} Empirically, I find the method to be fast and robust --- the examples folder shows that the algorithm solves a wide range of asset pricing models.
	\bibliography{bib}
	\bibliographystyle{aer}
\end{document}
